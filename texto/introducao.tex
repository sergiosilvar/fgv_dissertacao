% Faz com que o ínicio do capítulo sempre seja uma página ímpar
\cleardoublepage

% Inclui o cabeçalho definido no meta.tex
\pagestyle{fancy}

% Números das páginas em arábicos
\pagenumbering{arabic}

\chapter{Introdução}\label{intro}

Desde o anúncio da realização de dois dos principais eventos esportivos da era moderna na cidade do Rio de Janeiro, a Copa do Mundo em 2014 e as Olimpíadas de Verão em 2016, os preços dos imóveis residenciais e comerciais tiveram uma alta histórica. Por exemplo, o bairro do Leblon figura como o metro quadrado de mais alto valor do Brasil \nota{colocar referencia}, enquanto que outros bairros suburbanos também apresentam valores acima da média nacional. Em decorrência do aquecimento do mercado, vários empreendimentos imobiliários surgiram ao longo da cidade, tendo como a Barra da Tijuca e Jacarepaguá os bairros de maior concentração destes, em função da falta de espaço em locais mais tradicionais como aqueles na Zona Sul e Tijuca.

Entretanto, muitos destes empreendimentos apresentam preços considerados elevados, o que nos leva a perguntar como estimar o valor de um novo imóvel novo ou usado em um determinado local. No passado, pesquisas relacionadas a esse tema utilizavam uma ferramenta estatística conhecida como Modelo Hedônico  para a estimação de preço de um bem com base em suas partes constituintes. Em se tratando de imóveis, considerava-se suas partes constituintes como número de dormitórios, banheiros, suítes, existência de varanda, área construída, posição do apartamento em relação à rua, andar, vagas de garagem, entre outros. As observações eram então submetidas a um estimador, que na Matemática Aplicada é conhecido como Regressão Linear, e o resultado é um conjunto de coeficientes a serem aplicados em novas observações para a estimação do preço. Uma das limitações deste método é ausência da localização espacial das observações e, consequentemente, a carência de apreciação da autocorrelação espacial dos preços.

Nessa dissertação, lançamos mãos de técnicas de mineração de dados para uma análise exploratória dos preços dos imóveis na cidade do Rio de Janeiro, onde avaliaremos a assertividade da regressão linear com e sem localização espacial, e finalmente consideraremos os efeitos da autocorrelação espacial.  \nota{colocar ref} . 

