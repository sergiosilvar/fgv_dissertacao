% Faz com que o ínicio do capítulo sempre seja uma página ímpar
\cleardoublepage

% Inclui o cabeçalho definido no meta.tex
\pagestyle{fancy}

% Números das páginas em arábicos
\pagenumbering{arabic}

\chapter{Introdução}\label{intro}

 Com a conquista da cidade do Rio de Janeiro em sediar os Jogos Olímpicos de Verão de  2016 e participar como uma das sedes em outros eventos esportivos de grande porte como a Copa das Confederações de  2013 e a Copa do Mundo de 2014, a cidade do Rio de Janeiro viu-se como o foco de grandes investimentos de infraestrutura para mobilidade urbana e segurança pública, com impacto direto no mercado imobiliário, elevando o custo dos imóveis residenciais a um dos mais caros do país e despertando o interesse em avaliar a distribuição dos preços dos imóveis pela cidade a fim de investigar quais elementos são responsáveis pela determinação de preços dos imóveis.
 
 Uma das técnicas a respeito, Modelagem Hedônica, considera o bem em estudo, em nosso caso imóveis residenciais, como um bem constituído de diversas características de interesse dos consumidores onde cada uma contribui individualmente para o valor final. Para se obter a contribuição de cada característica, o preço do imóvel é atribuído como variável dependente e as características como variáveis independentes em um modelo de Regressão Linear e os coeficientes resultantes  determinam não apenas a contribuição de cada características para o preço final mas também permite observar a relevância relativa das características entre si. 

Embora existam alguns estudos similares no Brasil, a cidade do Rio de Janeiro é citada em apenas um,  \cite{Neto}, que utilizou  uma fonte de dados 120 observações de preços de imóveis, construindo 19 variáveis independentes para estimação da variável preço, obtidos a partir de anúncios de lançamentos de empreendimentos. Não há pesquisa sobre preços de imóveis residenciais usados, notadamente a maior parte das transações comerciais deste setor no município. 

Face ao exposto, o presente estudo objetiva abordar uma pesquisa exploratória a respeito da distribuição dos preços de imóveis residenciais na cidade do Rio de Janeiro, cujo principal objetivo é a construção de um modelo de estimação da variável preço para a cidade. Como objetivos auxiliares, procuramos determinar quais variáveis são relevantes para a determinação do preço e avaliar relações de correlação entre variáveis, como por exemplo, a relação entre a variável preço e distância a comunidades conhecidas como \textit{favelas}.

As justificativas para a discussão desse tema recaem sobre uma necessidade de revisão do modelo proposto originalmente por \cite{Neto} para a década atual em virtude do interesse dos agentes governamentais como termômetro de fenômenos sociais como crime, trânsito, oportunidades de emprego e constituição demográfica,  avaliação de investimentos em benfeitorias públicas e programas sociais, além dos interesses tributários como Imposto sobre a Propriedade Predial e Territorial Urbana, IPTU\footnote{IPTU: Constituição Federal , Título VI, Capítulo I, Seção V, artigo 156, inciso I.}, Imposto sobre Transmissão de \textit{Causa Mortis} e Doação de Bens ou Direitos, ITCMD\footnote{ITCMD: Constituição Federal , Título VI, Capítulo I, Seção IV, artigo 155, inciso II. } e Imposto de Transmissão de Bens Imóveis, ITBI\footnote{ITBI: Constituição Federal , Título VI, Capítulo I, Seção V, artigo 156, inciso II.}. A base de cálculo para os três impostos citados é o valor \textit{venal} do imóvel, uma estimativa promovida pelo agente público pois o valor efetivamente pago pelo comprador ao vendedor é desconhecido ao tributador e protegido por lei, a desejo do proprietário. Uma correta aferição desse valor permite ao poder público evitar tanto fraudes, quando o proprietário declara um valor menor que o pago, quanto abusos, quando o poder público determina um valor acima do pago, preservando-se o princípio da eficiência fiscal e tributária. Com relação ao setor privado, esse aborda Modelos Hedônicos em precificação de imóveis para o estudo de viabilidade de empreendimentos, determinação dos itens mais valorizados pelos consumidores e adequação de sua participação na intermediação de vendas de imóveis usados. O principal diferencial desse estudo aos demais realizados no Brasil é a obtenção dos dados de forma eletrônica, a partir de um serviço de classificados de imóveis online, a fim de obter um grande número de observações. 

Estruturamos o trabalho em XXXXX capítulos, organizados da seguinte forma:

O capítulo 1 apresenta a introdução ao trabalho, identificando o tema a ser abordado, uma breve descrição de trabalhos anteriores, os objetivos principal e auxiliares a serem alcançados, a metodologia a ser aplicada, as justificativas para escolha do tema e a organização do texto.

O capítulo 2 aborda uma revisão da literatura pertinente ao tema escolhido, descrevendo o modelo matemático, a descrição dos principais conceitos utilizados e trabalhos anteriores.

O capítulo 3 apresenta a metodologia utilizada com a descrição, obtenção e tratamento dos dados, aplicação do modelo matemático e discute o contexto.

O capítulo 4 discute os principais resultados conquistados.

O capítulo 5 aborda as conclusões que o estudo gerou e apresenta oportunidades de extensão do trabalho e propõe soluções que podem ser implementadas com tecnologia da informação.

A bibliografia á apresentada ao final do texto com as principais referências utilizadas pertinentes ao tema.

Uma série de apêndices conclui o trabalho, demonstrando imagens e tabelas em separado para melhor organização e impressão do estudo em papel.
